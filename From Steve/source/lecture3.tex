\documentclass[table]{beamer}

% Customize slide appearance
\mode<presentation>
{
  \usetheme{Warsaw}
  \setbeamercovered{transparent}
}

\usepackage[english]{babel}
\usepackage{times}
\usefonttheme[onlymath]{serif} 
\setbeamertemplate{navigation symbols}{}

% You can add any graphics to every slide by following command:
% \logo{\resizebox{0.1\textwidth}{!}{\includegraphics{col_small}}}
% \logo{\resizebox{0.2\textwidth}{!}{\includegraphics{imperialblue}}}

% Uncomment this, if you want the table of contents before each subsection.
% However, to edit slides in TeXWord avoiding this feature is good idea.
% \AtBeginSubsection[]
% {
%   \begin{frame}<beamer>
%     \frametitle{Outline}
%     \tableofcontents[currentsection,currentsubsection]
%   \end{frame}
% }

% If you wish to uncover everything in a step-wise fashion, uncomment
% the following command: 
%\beamerdefaultoverlayspecification{<+->}
\xdefinecolor{keyword}{rgb}{0,0,1}
\xdefinecolor{ctext}{rgb}{0.64,0.08,0.08}
\xdefinecolor{cline}{rgb}{0.17,0.57,0.69}
\xdefinecolor{comment}{rgb}{0,0.5,0.0}
\newif\ifschigh\schighfalse
\newcommand{\kw}[1]{\ifschigh\textcolor{red}{#1}\else\textcolor{keyword}{#1}\fi}
\newcommand{\kt}[1]{\ifschigh\textcolor{red}{#1}\else\textcolor{ctext}{#1}\fi}
\newcommand{\kc}[1]{\ifschigh\textcolor{red}{#1}\else\textcolor{comment}{#1}\fi}
\addtobeamertemplate{alerted text begin}{\global\schightrue%
}{}
\defbeamertemplate*{alerted text end}{default}{\global\schighfalse%
}



\newcounter{sckll}
\newcommand{\kr}{\setcounter{sckll}{1}}
\newcommand{\krr}[1]{\setcounter{sckll}{#1}}
\newcommand{\klvalue}{\ifnum\value{sckll}<10{\hphantom{0}}\fi\arabic{sckll}\addtocounter{sckll}{1}}
%\newcommand{\kl}{\ifschigh\textcolor{red}{\klvalue}\else\textcolor{cline}{\klvalue}\fi\hspace{2ex}}
\newcommand{\kl}{}

\let\oldurl=\url
\renewcommand{\url}[1]{\textcolor{blue}{\oldurl{#1}}}

\usepackage{tikz, pgfbaseplot, pgflibrarysnakes,pgflibraryarrows}

\begin{document}

% Title Data. We keep it after \begin{document} 
% to enable editing text in BaKoMa TeX Word.

\title[C for Science - Lecture 3]{Computing in C for Science}
\subtitle{Lecture 3 of 5}

% Use the \inst command to identify several affiliations.
\author[Steven Capper]{Dr. Steven Capper \\ {\tt steven.capper99@imperial.ac.uk}\\
\url{http://www2.imperial.ac.uk/~sdc99/ccourse/}}

\date{$30^\text{th}$ November 2011 }

\subject{C for Science} % Should be passed to PDF [YNI]
{
\logo{\includegraphics[width=0.30\textwidth]{imperialblue}}
\begin{frame}
  \titlepage
\end{frame}
}

\begin{frame}
\frametitle{Common Problems}
\begin{enumerate}
\item Misuse of Power!\\
$x^2$ should always be written {\tt x*x}, \emph{NOT} {\tt pow(x, 2.0)}.\\
$\sqrt{x}$ should be written {\tt sqrt(x)}, \emph{NOT} {\tt pow(x, 0.5)}.
\item Integer Division\\
Remember a fraction such as {\tt 1/3} is equal to zero. To get a floating point fraction, this should be rewritten {\tt 1.0/3.0}.
\end{enumerate}
\end{frame}

\begin{frame}
\frametitle{Input and Output in C - Streams}
\begin{itemize}
\item All I/O in C is accomplished via file \emph{streams}.
\item This stems from C's UNIX roots (every device is a file).
\item We have already seen {\tt printf}, formatted output to console.
\item C90 defines three streams which are initialised by default when a program starts.
\begin{center}
\rowcolors[]{1}{blue!20}{blue!10}
\begin{tabular}{l l}
{\tt stdin}&Input from the keyboard. (read only)\\
{\tt stdout}&Text console. (write only)\\
{\tt stderr}&Another text console. (write only)
\end{tabular}
\end{center}
\end{itemize}
\end{frame}

\begin{frame}
\frametitle{Why have {\tt stderr}?}
\begin{exampleblock}{Stream Redirection}
\begin{itemize}
\item UNIX (and now most other operating systems) allow for a C program's streams to be \emph{redirected} at the command line.
\item This is completely transparent to the C program (it just reads and writes data oblivious to the source).
\item One can redirect {\tt stdout} using ({\tt >}) as follows:
\begin{center}
\tt myprogram > output.txt
\end{center}
\item {\tt stderr} can be redirected using ({\tt 2>}):
\begin{center}
\tt myprogram 2> errorlog.txt
\end{center}
\item {\tt stdin} can be redirected using ({\tt <}):
\begin{center}
\tt myprogram < input.txt
\end{center}
\end{itemize}
\end{exampleblock}
Having a separate \emph{error stream}, allows for important messages to be filtered from data output.
\end{frame}


\begin{frame}[fragile]
\frametitle{{\tt printf} and {\tt scanf} - formatted output and input}
These are automatically connected to {\tt stdout} and {\tt stderr} respectively. The more generalised functions are {\tt fprintf} and {\tt fscanf}. As an example:
\begin{semiverbatim}
fprintf(stdout, \kt{"Text to stdout...\\n"});
fprintf(stderr, \kt{"Text to stderr...\\n"});
\end{semiverbatim}
We can also read text from a stream:
\begin{semiverbatim}
fscanf(stdin, \kt{"%lf"}, \&ptrDouble);
\end{semiverbatim}
\begin{itemize}
\item Additional streams can be opened using {\tt fopen}.
\end{itemize}
\end{frame}

\begin{frame}
\frametitle{{\tt fopen} - open a file stream}
\begin{semiverbatim}
\small
FILE * stream = fopen(\kw{char} * filename, \kw{char} * mode);
\end{semiverbatim}

Working backwards, {\tt mode} is a string telling C how to open the file:
{\footnotesize
\begin{center}
\rowcolors[]{1}{blue!20}{blue!10}
\begin{tabular}{l p{250pt}}
\tt mode& meaning\\
\tt "r"& Open {\tt filename} for reading. The file must exist or {\tt NULL} is returned.\\
\tt "w"& Open {\tt filename} for writing, starting from the beginning of the file. The file will be created if it doesn't already exist. Any old data will be overwritten.\\
\tt "a"& Open {\tt filename} for writing, starting from the end of the file. File will be \emph{appended} if it doesn't exist and created otherwise.\\
\tt "r+"& Open {\tt filename} for reading and writing, starting from the beginning. If the file doesn't exist, {\tt NULL} is returned.\\
\tt "w+"& Open {\tt filename} for reading and writing, starting from the beginning. If the file doesn't exist it's created.\\
\tt "a+"& Open {\tt filename} for reading and writing (append if exists).
\end{tabular}
\end{center}}
\end{frame}

\begin{frame}[fragile]
\frametitle{{\tt fopen} - II}
\begin{semiverbatim}
\small
FILE * stream = fopen(\kw{char} * filename, \kw{char} * mode);
\end{semiverbatim}
\begin{block}{\tt filename}
Must be a legal operating system file name. A safe option would be something similar to:
\vspace{-0.1in}
\begin{center}
\tt \kt{"results.txt"}
\end{center}
\end{block}

\begin{alertblock}{Absolute Filenames}
We can specify a full directory path to a filename:
\vspace{-0.2in}
\begin{semiverbatim}
\footnotesize
data = fopen(\kt{"C:\\\\TEMP\\\\mydata.dat"}, \kt{"w"}); \kc{/* Windows */}
data = fopen(\kt{"/tmp/mydata.dat"}, \kt{"w"});      \kc{/* UNIX */}
\end{semiverbatim}
\end{alertblock}

\begin{exampleblock}{Relative Filenames}
It is much safer to omit the full path:
\vspace{-0.2in}
\begin{semiverbatim}
\footnotesize
data = fopen(\kt{"mydata.dat"}, \kt{"w"}); \kc{/* works on most os' */}
\end{semiverbatim}
\end{exampleblock}
\end{frame}

\begin{frame}[fragile]
\frametitle{{\tt fopen} - III}
\begin{semiverbatim}
\small
FILE * stream = fopen(\kw{char} * filename, \kw{char} * mode);
\end{semiverbatim}
\vspace{-0.1in}
\begin{itemize}
\item Here {\tt stream} is a pointer to a {\tt FILE} data type.
\item {\tt stdin}, {\tt stdout} and {\tt stderr} are also pointers (which are opened automatically for us).
\item Sometimes it is not possible to open a file (it may not exist, or belong to another user etc...), in this case a special pointer value {\tt NULL} (which is equal to zero) is returned.
\end{itemize}

\begin{block}{{\tt NULL} pointers}
A {\tt NULL} pointer is a special value, usually signalling failure. This can be checked for:
\vspace{-0.2in}
\begin{semiverbatim}
\footnotesize
stream = fopen(\kt{"illegal/\\\\filename.txt"}, \kt{"w"});
if (!stream)
\{
   fprintf(stderr, \kt{"Unable to open file\\n"});
   ...
\end{semiverbatim}
\end{block}
\end{frame}

\begin{frame}
\frametitle{{\tt fclose} - Closing a file stream}
\begin{itemize}
\item Any file streams {\tt fopen}'ed should be closed using:
\begin{center}
\tt fclose(stream);
\end{center}
\begin{itemize}
\item File read/writes may go to an internal buffer before they go to the disk (to speed things up).
\item An explicit file close ensures that any pending data is written to disk.
\end{itemize}
\item For the lazy there is:
\begin{center}
\tt fcloseall();
\end{center}
\item Do \emph{NOT} call {\tt fclose} on any of the following:
\begin{center}
{\tt stdin, stdout} \& \tt stderr
\end{center}
\end{itemize}
\end{frame}

\begin{frame}
\frametitle{Moving around a file - {\tt rewind}, {\tt ftell} and {\tt fseek}}
\begin{block}{\tt rewind(stream);}
Start the next read/write from the beginning of the file.
\end{block}

\begin{block}{\tt fpos = ftell(stream);}
Get an \kw{\tt int} telling us how far we are in the file (from the beginning).
\end{block}

\begin{block}{\tt fseek(stream, fpos, SEEK\_SET);}
Move {\tt fpos} bytes into the file (counted from the beginning).
\end{block}

\begin{block}{\tt fseek(stream, fpos, SEEK\_CUR);}
Move {\tt fpos} bytes forward into file (from current position).
\end{block}

\begin{block}{\tt fseek(stream, fpos, SEEK\_END);}
Move {\tt fpos} bytes from the end of the file.
\end{block}
\end{frame}

\begin{frame}
\frametitle{File Types}
Files in C can contain data in two formats:
\begin{enumerate}
\item As text, where each byte is interpreted as an ASCII character mapping.
\item Or as raw \emph{binary} data, the meaning of which is left to the program to determine.
\end{enumerate}

These files are referred to as text files and binary files respectively. All the I/O covered so far applies to text files.

\begin{block}{}
\begin{center}
Each format has it's advantages and disadvantages.
\end{center}
\end{block}
\end{frame}

\begin{frame}
\frametitle{Text versus Binary Files}
\begin{block}{Text Files}
\begin{itemize}
{\setbeamercolor{local structure}{parent=example text}
\item \textcolor{green}{Pro} - Can be read by almost any program on almost any computer. (And by humans!).
}
{\setbeamercolor{local structure}{parent=alerted text}
\item \textcolor{red}{Con} - Inefficient - Numbers can take up as much as twice the space as necessary.}
\end{itemize}
\end{block}

\begin{block}{Binary Files}
\begin{itemize}
{\setbeamercolor{local structure}{parent=example text}
\item \textcolor{green}{Pro} - Accurate and efficient.
}
{\setbeamercolor{local structure}{parent=alerted text}
\item \textcolor{red}{Con} - Incompatible - Binary data can be very difficult to read by other programs, and incredibly difficult between processor architectures.}
\end{itemize}
\end{block}
\end{frame}

\begin{frame}[fragile]
\frametitle{Handling Binary Files}
Binary files are opened using {\tt fopen}, we need to add {\tt "b"} to the file mode though:
\begin{semiverbatim}
stream = fopen(\kt{"data.dat"}, \kt{"rb"});
\end{semiverbatim}
Data is read from and written to a binary file using {\tt fread} and {\tt fwrite}. For example:
\begin{semiverbatim}
\kw{double} values[10] = \{0.0\};
\kc{/* we have "stream" an open binary file */}
read = fread(values, \kw{sizeof}(\kw{double}), 10, stream);
\kw{if} (read != 10)
\{ \emph{read failed} \}
\end{semiverbatim}
\begin{block}{}
\begin{center}
Please see \kt{\tt <stdio.h>} for more details.
\end{center}
\end{block}
\end{frame}

\begin{frame}[fragile]
\frametitle{Permanent versus Temporary Files}
\begin{itemize}
\item Sometimes we may not wish for a file to be permanent.
\item Temporary files provide a means for storing intermediate computations.
\item We open a temporary file using the {\tt tmpfile()} function.
\end{itemize}
\begin{block}{Example}
\begin{semiverbatim}
stream = tmpfile();
\kw{if} (!stream)
\{
\kc{/* handle failure */}
\}
\kc{/* fread, fwrite, fprintf, fscanf */}
fclose(stream);
\end{semiverbatim}
\end{block}
\end{frame}

\begin{frame}[fragile]
\frametitle{Pointers and Arrays}
We saw the connection between arrays and pointers in the last lecture. Given the array:
\begin{center}
\tt \kw{double} ar [4] = \{1.0, 2.0, 3.0, 4.0\};
\end{center}
The elements of this array can be accessed as follows:
\begin{semiverbatim}
      *(ar) == ar[0] == 1.0
    *(ar+1) == ar[1] == 2.0
    *(ar+2) == ar[2] == 3.0
\end{semiverbatim}
Pointer referencing is also supported:
\begin{semiverbatim}
        ar == \&ar[0]
    (ar+1) == \&ar[1]
    (ar+2) == \&ar[2]
\end{semiverbatim}
\end{frame}

\begin{frame}
\frametitle{Allowed Pointer Operations}
\begin{tabular}{l l}
Declaration:&{\tt \kw{double} * pA, * pB;}\\
Assignment:&{\tt pA = \&var;}\\
Increment:&{\tt pA = pA + 1;}\\
Decrement:&{\tt pA = pA - 1;}\\
Difference:&{\tt gap = pA - pB;}\\
Comparison:&{\tt \kw{if}(pA == pB)}\\
De-referencing:&{\tt *pA = val;}
\end{tabular}
\end{frame}

\begin{frame}[fragile]
\frametitle{Fixed Size Two-Dimensional Arrays}
We can declare arrays of dimension higher than one, as follows:
\begin{semiverbatim}
\kw{double} a[2][3] = \{\{1.0, 2.0, 3.0\},
                  \{2.0, 3.0, 4.0\}\};
\end{semiverbatim}
Where the elements of {\tt a} are denoted as expected:
\begin{center}
\begin{tabular}{|c|c|c|}
\hline
\tt a[0][0]&\tt a[0][1]&\tt a[0][2]\\
\hline
\tt a[1][0]&\tt a[1][1]&\tt a[1][2]\\
\hline
\end{tabular}
\end{center}
To access the top left element:
\begin{semiverbatim}
myVal = a[0][0]; \kc{/* equal to 1.0 */}
\end{semiverbatim}
\end{frame}

\begin{frame}[fragile]
\frametitle{Fixed Size Two-Dimensional Arrays II}
\vspace{-0.2in}
\begin{semiverbatim}
\footnotesize
\kr\kl\kw{\#include} \kt{<stdio.h>}
\kl\kw{\#define} COLS 3
\kl
\kl\kw{void} printArray(\kw{int} matrix[][COLS], int rows)
\kl\{
\kl   \kw{int} i, j;
\kl   \kw{for} (i = 0; i < rows; i++)
\kl   \{
\kl      \kw{for} (j = 0; j < COLS; j++)
\kl         printf(\kt{"\%d "}, matrix[i][j]);    
\kl      printf(\kt{"\\n"});
\kl   \}
\kl\}
\kl
\kl\kw{int} main()
\kl\{
\kl   \kw{int} matrix[2][COLS] = \{\{1, 2, 3\},\{4, 5, 6\}\};
\kl   printArray(matrix, 2);
\kl   \kw{return} 0;
\kl\}
\end{semiverbatim}
\end{frame}

\begin{frame}
\frametitle{Fixed Size Two-Dimensional Arrays III}
We have seen an example of a two dimensional array:
\begin{center}
%\rowcolors[]{1}{blue!20}{blue!10}
\begin{tabular}{|c|c|c|}
\hline
\tt a[0][0]&\tt a[0][1]&\tt a[0][2]\\
\hline
\tt a[1][0]&\tt a[1][1]&\tt a[1][2]\\
\hline
\end{tabular}
\end{center}
\begin{itemize}
\item In memory it is arranged as follows:
\resizebox{0.9\textwidth}{!}{
\begin{tabular}{|c|c|c|c|c|c|}
\hline
\tt a[0][0]&\tt a[0][1]&\tt a[0][2]&
\tt a[1][0]&\tt a[1][1]&\tt a[1][2]\\
\hline
\end{tabular}}
(i.e. as a one dimensional array).
\item Fixed size arrays are very inflexible as they require the dimensions to be ``hard coded''.
\item They are allocated from the stack thus large arrays may cause problems.
\item \emph{Dynamically allocated} arrays overcome these restrictions.
\end{itemize}
\end{frame}

\begin{frame}
\frametitle{Dynamic Allocation}
\begin{block}{The heap}
\begin{itemize}
\item C has set aside a special block of memory known as the \emph{heap}.
\item Memory can be requested from the heap.
\item If memory is available, it is provided, otherwise a request is made to the operating system.
\item The implementation details of the heap are ``undefined''.
\end{itemize}
\end{block}
\begin{block}{\tt sizeof}
\begin{itemize}
\item We need to know how much memory to request.
\item The \kw{\tt sizeof} keyword returns the size (in bytes), of a variable or data type.
\end{itemize}
\end{block}
\end{frame}

\begin{frame}[fragile]
\frametitle{How to Borrow Memory}
We use the {\tt malloc} function (defined in \kt{\tt <stdlib.h>}) to request memory and {\tt free} to return it to the heap.
\vspace{-0.1in}
\begin{semiverbatim}
\scriptsize
\kr\kl\kw{\#include} \kt{<stdio.h>}
\kl\kw{\#include} \kt{<stdlib.h>}
\kl\kw{\#define} NCOUNT 10
\kl\kw{int} main()
\kl\{
\kl   \kw{int} * numbers, loop;
\kl   numbers = (\kw{int} *) malloc(\kw{sizeof}(\kw{int})*NCOUNT);
\kl   \kw{if} (!numbers)
\kl   \{
\kl      fprintf(stderr, \kt{"Unable to allocate memory\\n"});
\kl      return -1;
\kl   \}
\kl   \kw{for} (loop = 0; loop < NCOUNT; loop++)
\kl      numbers[loop] = loop;
\kl
\kl   \kw{for} (loop = 0; loop < NCOUNT; loop++)
\kl      printf(\kt{"numbers[\%d] = \%d\\n"}, loop, numbers[loop]);
\kl
\kl   \kw{free}(numbers);
\kl   \kw{return} 0;
\kl\}
\end{semiverbatim}
\end{frame}

\begin{frame}[fragile]
\frametitle{How to Borrow Memory - Details}
\begin{semiverbatim}
      \kw{void} * malloc(size_t Size);
      \kw{void} free(\kw{void} * Memory);
\end{semiverbatim}
\begin{block}{What is {\tt void *}?}
\begin{itemize}
\item {\tt \kw{void} *} is a pointer to an unknown data type, it consists only of an address.
\item We can explicitly cast from {\tt \kw{void} *} to any other pointer type.
\item Any pointer type can be implicitly casted to a {\tt \kw{void} *}.
\item A value of {\tt NULL} usually means that a function encountered an error.
\end{itemize}
\end{block}

\begin{block}{What is {\tt size\_t}?}
{\tt size\_t} is defined to be an unsigned integer type large enough to hold numbers returned by {\tt \kw{sizeof}}.
\end{block}
\end{frame}

\begin{frame}
\frametitle{How to Borrow Memory - More Details}
Two more popular memory management functions are:
\begin{center}
\rowcolors[]{1}{blue!20}{blue!10}
\begin{tabular}{l l}
\tt realloc&Modifies the size of a memory block.\\
\tt calloc&Allocates memory, and sets every byte to 0.\\
\end{tabular}
\end{center}
\begin{alertblock}{Some common heap problems}
\begin{itemize}
\item Allocating memory is slow and should be done as few times as possible.
\item Memory allocation can fail, every {\tt malloc}/{\tt calloc} should be checked (or at the very least {\tt assert}ed).
\item Every {\tt malloc}/{\tt calloc} should have a corresponding {\tt free}, memory is \emph{leaked} if it's not {\tt free}d after use.
\end{itemize}
\end{alertblock}
\end{frame}

\begin{frame}[fragile]
\frametitle{1-D Example: Fibonacci (again!)}
\begin{semiverbatim}
\tiny
\kr\kl\kw{\#include} \kt{<stdlib.h>}
\kl\kw{\#include} \kt{<stdio.h>}
\kl
\kl\kw{int} main()
\kl\{
\kl   \kw{double} * fibs;
\kl   \kw{unsigned int} nfibs=0, loop;
\kl   \kw{while} (nfibs < 2)
\kl   \{
\kl      printf(\kt{"How many Fibonacci numbers are needed (>1)?\\n"});
\kl      scanf(\kt{"\%u"}, \&nfibs);
\kl   \}
\kl   
\kl   fibs = (\kw{double} *) malloc(\kw{sizeof}(\kw{double})*nfibs);
\kl   \kw{if} (!fibs) \kc{/* malloc failed? */}
\kl   \{
\kl      fprintf(stderr, \kt{"Unable to allocate memory!\\n"});
\kl      \kw{return} -1;
\kl   \}       
\kl   
\kl   fibs[0] = 1.0; fibs[1] = 1.0;
\kl   \kw{for} (loop = 2; loop < nfibs; loop++)
\kl      fibs[loop] = fibs[loop-1]+fibs[loop-2];
\kl      
\kl   \kw{for} (loop = 0; loop < nfibs; loop++)
\kl      printf(\kt{"fib[\%u] = \%lg\\n"}, loop, fibs[loop]);
\kl      
\kl   free(fibs);
\kl   \kw{return} 0;
\kl\}
\end{semiverbatim}
\end{frame}

{
\setbeamercolor{frametitle}{bg=red}
\begin{frame}
\frametitle{Pointers to Pointers: {\tt **ptr}}
\begin{itemize}
\item Pointers in C are very flexible, to the extent that we can form pointers to pointers!
\item This is known more formally as the ``level of indirection''.
\item Tensors of arbitrary rank (Fortran 95 is limited to 7) can be formed easily in C.
\item The most useful example in C being matrices.
\end{itemize}
\begin{block}{Matrices in C}
Matrices in C are commonly represented by the {\tt \kw{double} **} type. This means ``a pointer to a pointer to a {\tt \kw{double}}''.
\end{block}
\end{frame}
}

\begin{frame}[fragile]
\frametitle{Constructing Matrices with Pointers}
\vspace{-0.2in}
\begin{semiverbatim}
\footnotesize
\kr\kl\kw{double} ** makeMatrix(\kw{unsigned int} rows, \kw{unsigned int} cols)
\kl\{
\kl   \kw{unsigned int} i;
\kl   \kw{double} ** matrix;
\kl        
\kl   matrix = (\kw{double} **) malloc(rows * \kw{sizeof}(\kw{double} *));
\kl   \kw{if} (!matrix) \kw{return} NULL; \kc{/* failed */}
\kl
\kl   \kw{for} (i = 0; i < rows; i++)
\kl   \{
\kl      matrix[i] = (\kw{double} *) malloc(cols*\kw{sizeof}(\kw{double}));
\kl      \kw{if} (!matrix[i])
\kl         \kw{return} NULL; \kc{/* lazy, we should really free} 
\kl                         \kc{all the memory allocated above */}
\kl   \}
\kl
\kl   \kw{return} matrix;
\kl\}
\end{semiverbatim}
\end{frame}

\begin{frame}[fragile]
\frametitle{Accessing Matrix Elements}
\begin{block}{Usage pattern for {\tt makeMatrix}}
\begin{semiverbatim}
\kw{double} ** matrix = makeMatrix(rows, cols);
\kw{for} (i=0; i < rows; i++)
   \kw{for} (j=0; j < cols; j++)
      matrix[i][j] = 0.0;
\emph{free the matrix}      
\end{semiverbatim}
\end{block}
\begin{itemize}
\item Accessing the dynamically allocated array looks identical to the fixed size ones, but ``under the hood'' things are a little different:
\begin{alertblock}{}
\begin{center}
\tt matrix[row][col] = *(*(matrix + row) + col)
\end{center}
\end{alertblock}
\item The {\tt makeMatrix} code on the previous slide contained a lot of {\tt malloc} statements, is there a better way to allocate a matrix? (yes!)
\end{itemize}
\end{frame}

{
\setbeamercolor{frametitle}{bg=red}
\begin{frame}[fragile]
\frametitle{A Better Way of Allocating Matrices}
\begin{semiverbatim}
\scriptsize
\kr\kl\kw{double} ** allocMatrix(\kw{unsigned int} rows, \kw{unsigned int} cols)
\kl\{
\kl   \kw{double} ** matrix;
\kl   \kw{unsigned int} i;
\kl
\kl   matrix = (\kw{double} **) malloc (rows*sizeof(\kw{double} *));
\kl   \kw{if} (!matrix) \kw{return} NULL; \kc{/* failed */}
\kl
\kl   matrix[0] = (\kw{double} *) malloc (rows*cols*\kw{sizeof}(\kw{double}));
\kl   \kw{if} (!matrix[0])
\kl   \{
\kl      free(matrix); \kc{/* we don't need matrix any more */}
\kl      \kw{return} NULL;  \kc{/* failed */}
\kl   \}
\kl
\kl   \kw{for} (i = 1; i < rows; i++)
\kl      matrix[i] = matrix[i-1] + cols;
\kl
\kl   \kw{return} matrix;
\kl\}
\end{semiverbatim}
\end{frame}
}

\begin{frame}[fragile]
\frametitle{Why is {\tt allocMatrix} Better?}
\begin{itemize}
\item {\tt allocMatrix} only uses 2 {\tt malloc}s whilst, {\tt makeMatrix} uses {\tt cols + 1}.
\item Meaning there are fewer points of failure (we only check two pointers for {\tt NULL}),
\item It is much easier to free a matrix allocated with the {\tt allocMatrix} function, all we need to do is:
\begin{semiverbatim}
\kw{void} freeMatrix(\kw{double} ** matrix)
\{
   free(matrix[0]);
   free(matrix);
\}
\end{semiverbatim}
\end{itemize}
\end{frame}

\begin{frame}
\frametitle{Case Study: Matrix Addition}
Let's define some utility functions to:
\begin{enumerate}
\item Allocate memory for the matrix ({\tt allocMatrix}) - {\bf done},
\item Free a matrix ({\tt freeMatrix}) - {\bf done},
\item Print a matrix ({\tt printMatrix}),
\item Create a random matrix ({\tt randomMatrix}),
\item Add matrices together ({\tt addMatrices})
\end{enumerate}
We drive all these functions using a {\tt main} function.
\end{frame}

\begin{frame}[fragile]
\frametitle{{\tt printMatrix} and {\tt randomMatrix}}
\begin{semiverbatim}
\scriptsize
\kr\kl\kw{void} printMatrix(\kw{double} ** matrix, \kw{unsigned int} rows,
\kl                                   \kw{unsigned int} cols)
\kl\{
\kl   \kw{unsigned int} i, j;
\kl
\kl   \kw{for} (i = 0; i < rows; i++)
\kl   \{
\kl      \kw{for} (j = 0; j < cols; j++)
\kl         printf(\kt{"\%8.5lf "}, matrix[i][j]);
\kl      printf(\kt{"\\n"});
\kl   \}
\kl\}
\end{semiverbatim}

\begin{semiverbatim}
\scriptsize
\kr\kl\kw{void} randomMatrix(\kw{double} ** matrix, \kw{unsigned int} rows,
\kl                                    \kw{unsigned int} cols)
\kl\{
\kl   \kw{unsigned int} i, j;
\kl   \kw{for} (i = 0; i < rows; i++)
\kl      \kw{for} (j = 0; j < cols; j++)
\kl         matrix[i][j] = (\kw{double})rand()/RAND_MAX;
\kl\}
\end{semiverbatim}
\end{frame}

\begin{frame}[fragile]
\frametitle{{\tt addMatrices}}
\begin{semiverbatim}
\scriptsize
\kr\kl\kw{void} addMatrices(\kw{double} ** matrixA, \kw{double} ** matrixB,
\kl                 \kw{double} ** matrixR,
\kl                 \kw{unsigned int} rows, \kw{unsigned int} cols)
\kl\{
\kl   \kw{unsigned int} i, j;
\kl   \kw{for} (i = 0; i < rows; i++)
\kl      \kw{for} (j = 0; j < cols; j++)
\kl         matrixR[i][j] = matrixA[i][j]+matrixB[i][j];
\kl\}
\end{semiverbatim}
\end{frame}

\begin{frame}[fragile]
\frametitle{The {\tt main} function}
\begin{semiverbatim}
\tiny
\kr\kl\kw{int} main()
\kl\{
\kl   \kw{unsigned int} rows, cols;
\kl   \kw{double} ** matrixA, ** matrixB, **matrixC;
\kl   printf(\kt{"Enter rows cols: "});
\kl   scanf(\kt{"%u %u"}, &rows, &cols);
\kl
\kl   matrixA = allocMatrix(rows, cols);
\kl   matrixB = allocMatrix(rows, cols);
\kl   matrixC = allocMatrix(rows, cols);
\kl
\kl   \kw{if} (!matrixA || !matrixB || !matrixC)
\kl   \{   \kc{/* a little lazy, but it does the job */}
\kl      fprintf(stderr, \kt{"Unable to allocate matrices!\\n"});
\kl      \kw{return} -1;
\kl   \}
\kl
\kl   randomMatrix(matrixA, rows, cols); randomMatrix(matrixB, rows, cols);
\kl   addMatrices(matrixA, matrixB, matrixC, rows, cols);
\kl
\kl   printf(\kt{"\\n\\nmatrix A = \\n"});
\kl   printMatrix(matrixA, rows, cols);
\kl   printf(\kt{"\\n\\nmatrixB = \\n"});
\kl   printMatrix(matrixB, rows, cols);
\kl   printf(\kt{"\\n\\nmatrixA + matrixB = \\n"});
\kl   printMatrix(matrixC, rows, cols);
\kl
\kl   freeMatrix(matrixC); freeMatrix(matrixB); freeMatrix(matrixA);
\kl   \kw{return} 0;
\kl\}
\end{semiverbatim}
\end{frame}

\begin{frame}[fragile]
\frametitle{Results}
\vspace{-0.2in}
\begin{semiverbatim}
\scriptsize
Enter rows cols: 4 4


matrix A =
 0.00125  0.56359  0.19330  0.80874
 0.58501  0.47987  0.35029  0.89596
 0.82284  0.74660  0.17411  0.85894
 0.71050  0.51353  0.30399  0.01498


matrixB =
 0.09140  0.36445  0.14731  0.16590
 0.98853  0.44569  0.11908  0.00467
 0.00891  0.37788  0.53166  0.57118
 0.60176  0.60717  0.16623  0.66305


matrixA + matrixB =
 0.09265  0.92804  0.34062  0.97464
 1.57353  0.92557  0.46937  0.90063
 0.83175  1.12449  0.70577  1.43013
 1.31227  1.12070  0.47023  0.67803
\end{semiverbatim}
\end{frame}

\end{document}

