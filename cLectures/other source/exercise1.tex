\documentclass[a4paper,12pt]{article}
\usepackage{amsmath, url}
        %%%%%%%%%%%%%%%%%%%%%%%%%%%%%%
        \oddsidemargin  -0.5in
        \evensidemargin -0.5in
        \textwidth      7.0in
        \headheight     0.0in
        \topmargin      -1.0in
        \textheight=10.5in
        %%%%%%%%%%%%%%%%%%%%%%%%%%%%%%

\begin{document}
\begin{center}
\large C for Science - Practical Exercise \#1
\end{center}

\begin{enumerate}
\item Open Visual Studio (or similar) and set up as per the handout. \\
Type in the following program:
\begin{verbatim}
#include <stdio.h>
int main(void)
{
   printf("Hello World!");
   return 0;
}
\end{verbatim}
Compile and run it.
\begin{enumerate}
\item Replace the space between \verb|"Hello| and \verb|World!"| in turn with each of the 7 characters: \verb|\n|, \verb|\t|, \verb|\b|, \verb|\f|, \verb|\\|, \verb|\"| and \verb|\'|. For each case, re-compile and run the program.
What happens?
\end{enumerate}

\item Type in the following program:
\begin{verbatim}
#include <stdio.h>
int main(void) {
   int i1=1, i2=2222, i3=333333333;
   float x1=1.0, x2=3.1415926, x3 = -1.e-10;
   printf("i1, i2, i3 = %d %d %d\n", i1, i2, i3);
   printf("x1, x2, x3 = %f %f %f\n", x1, x2, x3);
   return 0;
}
\end{verbatim}
Compile and run it. How does the output differ to what is expected? What do you notice when making the following changes:
\begin{enumerate}
\item Change \verb|%d %d %d| in line 6 to \verb|%6d %6d %6d|.
\item Change \verb|%f %f %f| in line 7 to \verb|%e %e %e|.
\item Change \verb|%f %f %f| in line 7 to \verb|%g %g %g|.
\item Change \verb|%f %f %f| in line 7 to \verb|%10.2g %10.4g %10.6g| (A warning: the exact output generated from the format descriptor \verb|%g| can vary from compiler to compiler!)
\end{enumerate}

\item {\bf Integer Arithmetic:} Type in the following program:
\begin{verbatim}
#include <stdio.h>
int main(void) {
   short i1=11, i2=22, i3, i4, i5, i6;
   i3 = i1 + i2;
   i4 = i1 * i2;
   i5 = i1 / i2;
   i6 = i2 / i1;
   printf("i3 to i6 = %d %d %d %d\n", i3, i4, i5, i6);
   return 0;
}
\end{verbatim}
Compile and run it. If you change line 4 such that \verb|i1=11111| and \verb|i2=22222|, does the program produce what you expect? (Hint: what is $33333-2^{16}$)?

\item {\bf Floating Point Arithmetic:} Type in the following program:
\begin{verbatim}
#include <stdio.h>
int main(void)
{
   float x1=1.0e5, x2=3.1415926, x3, x4, x5, x6;
   x3 = x1 + x2;
   x4 = x1 * x2;
   x5 = x1 / x2;
   x6 = x2 / x1;
   printf("x3 - x6 = %f %f %f %f\n", x3, x4, x5, x6);
   return 0;
}
\end{verbatim}
Compile and run it.
\begin{enumerate}
\item Change line 4 such that \verb|x1=1.0e25| and \verb|x2=3.0e10|. Does this give the result you expect?
\item Change the operator in line 5 such that \verb|x3=x1 % x2|, why does this not compile? (In C \verb|%| is the modulo operator).
\end{enumerate}

\item Type in the following program:
\begin{verbatim}
#include <stdio.h>
int main(void)
{
   int low = -40, high = 140, step = 5, f, c = low;
   while (c <= high)
   {
      f = 32+9*c/5;
      printf("%6d \t %6d\n", c, f);
      c = c + step;
   }
   return 0;
}
\end{verbatim}
Compile and run it. Edit the program so that it calculated {\tt f} correctly for {\tt c} ranging from -40 to 140 in steps of 2. (\emph{What type do the numbers have to be?}). Amend the {\tt printf} so that no decimal places are printed for {\tt c} and one decimal place is printed for the values of {\tt f}. Add a line of code to print a heading at the beginning of the printed table.

\item The code fragment
\begin{verbatim}
#include <stdio.h>
int main(void)
{
   double a0, a1;
   printf("Enter coefficients of Linear Equation a1*x + a0 = 0\n"
          "in the order a1, a0, separated by spaces:");
   scanf("%lf %lf", &a1, &a0);
}
\end{verbatim}
will prompt the user for two numbers and read them into the variables {\tt a1} and {\tt a0}. Finish the program by adding code to solve the linear equation and display the result on screen. Ensure that the case {\tt a1 == 0.0} is dealt with properly.
\end{enumerate}
\end{document}
